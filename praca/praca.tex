%
% Niniejszy plik stanowi przyk³ad formatowania pracy magisterskiej na
% Wydziale MIM UW.  Szkielet u¿ytych poleceñ mo¿na wykorzystywaæ do
% woli, np. formatujac wlasna prace.
%
% Zawartosc merytoryczna stanowi oryginalnosiagniecie
% naukowosciowe Marcina Wolinskiego.  Wszelkie prawa zastrze¿one.
%
% Copyright (c) 2001 by Marcin Woliñski <M.Wolinski@gust.org.pl>
% Poprawki spowodowane zmianami przepisów - Marcin Szczuka, 1.10.2004
% Poprawki spowodowane zmianami przepisow i ujednolicenie 
% - Seweryn Kar³owicz, 05.05.2006
% dodaj opcjê [licencjacka] dla pracy licencjackiej
\documentclass{pracamgr}

\usepackage{polski}

%Jesli uzywasz kodowania polskich znakow ISO-8859-2 nastepna linia powinna byc 
%odkomentowana
\usepackage[latin2]{inputenc}
%Jesli uzywasz kodowania polskich znakow CP-1250 to ta linia powinna byc 
%odkomentowana
%\usepackage[cp1250]{inputenc}

% Dane magistranta:

\author{Julia Herman-Iżycka, Marcin Jedynak, Jakub Partyka, Aleksandra Skrzypczak}

\nralbumu{280468, 292629, 236188, 234586}

\title{Google Apps dla MIMUW}

\tytulang{Google Apps for MIMUW}

%kierunek: Matematyka, Informatyka, ...
\kierunek{Informatyka}

% informatyka - nie okreslamy zakresu (opcja zakomentowana)
% matematyka - zakres moze pozostac nieokreslony,
% a jesli ma byc okreslony dla pracy mgr,
% to przyjmuje jedna z wartosci:
% {metod matematycznych w finansach}
% {metod matematycznych w ubezpieczeniach}
% {matematyki stosowanej}
% {nauczania matematyki}
% Dla pracy licencjackiej mamy natomiast
% mozliwosc wpisania takiej wartosci zakresu:
% {Jednoczesnych Studiow Ekonomiczno--Matematycznych}

% \zakres{Tu wpisac, jesli trzeba, jedna z opcji podanych wyzej}

% Praca wykonana pod kierunkiem:
% (podaæ tytu³/stopieñ imiê i nazwisko opiekuna
% Instytut
% ew. Wydzia³ ew. Uczelnia (je¿eli nie MIM UW))
\opiekun{dra Jacka Sroki\\
  Instytut Informatyki\\
  }

% miesi±c i~rok:
\date{Maj 2012}

%Podaæ dziedzinê wg klasyfikacji Socrates-Erasmus:
\dziedzina{ 
%11.0 Matematyka, Informatyka:\\ 
%11.1 Matematyka\\ 
%11.2 Statystyka\\ 
11.3 Informatyka\\ 
%11.4 Sztuczna inteligencja\\ 
%11.5 Nauki aktuarialne\\
%11.9 Inne nauki matematyczne i informatyczne
}

%Klasyfikacja tematyczna wedlug AMS (matematyka) lub ACM (informatyka)
\klasyfikacja{H.3.5 Online information services\\

}

% S³owa kluczowe:
\keywords{GoogleApps, platforma edukacyjna, e-learning, Moodle}

% Tu jest dobre miejsce na Twoje w³asne makra i~¶rodowiska:
\newtheorem{defi}{Definicja}[section]

% koniec definicji

\begin{document}
\maketitle

%tu idzie streszczenie na strone poczatkowa
\begin{abstract}
W ramach projektu stworzono zestaw narzędzi usprawniających korzystanie 
z serwisów internetowych Uniwersytetu Warszawskiego. Tworzona aplikacja zapewni integrację 
technologii Google z danymi udostępnianymi przez USOS. Grono potencjalnych odbiorców stanowią 
studenci i pracownicy akademiccy. Ważnym elementem systemu będzie zbiór rozwiązań 
alternatywnych w stosunku do funkcjonującej platformy Moodle. Nie stworzono dotychczas 
produktu zapewniającego wszystkie pożądane funkcje, co stanowi motywację do realizacji projektu. 
\end{abstract}

\tableofcontents
%\listoffigures
%\listoftables

\chapter*{Wprowadzenie}
\addcontentsline{toc}{chapter}{Wprowadzenie}

\section{Stworzone oprogramowanie}
\section{Motywacja}
%dlaczego powstało takie oprogramowanie (frakmenty wizji)
\section{Zawartość pracy}
%co opisalismy w pracy licencjackiej

\chapter{Sformułowanie problemu}
\section{Definicje}
%definicje z wizji - co to jest Moodle, e-learning
\section{Charakterystyka użytkowników}
%też z wizji - co robi pracownik a co student

\chapter{Funkcjonalność}
\section{Dostepne funkcje}
%tu opis co można zrobić przy użyciu naszego systemu

\section{Przypadki użycia}
%tu chcemy opisać przykładowy przypadek użycia systemu - cos podobnego 
%do opisanego wcześniej przypadku użycia




\chapter{Zastosowane technologie}
%dlaczego użylismy tych technologii i czym się charakteryzują
\section{Google Web Toolkit}
\section{Google App Engine}

\chapter{Podsumowanie}
\section{Dokumentacja}
%co napisaliśmy poza pracą licencjacką
\section{Możliwe rozszerzenia}
%jak dalej może rozwijać się produkt



\begin{thebibliography}{99}
\addcontentsline{toc}{chapter}{Bibliografia}


%Przykład wpisu do bibliografii
%\bibitem[Bea65]{beaman} Juliusz Beaman, \textit{Morbidity of the Jolly
%    function}, Mathematica Absurdica, 117 (1965) 338--9.



\end{thebibliography}

\end{document}


%%% Local Variables:
%%% mode: latex
%%% TeX-master: t
%%% coding: latin-2
%%% End:
