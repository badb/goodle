%
% Niniejszy plik stanowi przyk³ad formatowania pracy magisterskiej na
% Wydziale MIM UW.  Szkielet u¿ytych poleceñ mo¿na wykorzystywaæ do
% woli, np. formatujac wlasna prace.
%
% Zawartosc merytoryczna stanowi oryginalnosiagniecie
% naukowosciowe Marcina Wolinskiego.  Wszelkie prawa zastrze¿one.
%
% Copyright (c) 2001 by Marcin Woliñski <M.Wolinski@gust.org.pl>
% Poprawki spowodowane zmianami przepisów - Marcin Szczuka, 1.10.2004
% Poprawki spowodowane zmianami przepisow i ujednolicenie 
% - Seweryn Kar³owicz, 05.05.2006
% dodaj opcjê [licencjacka] dla pracy licencjackiej
\documentclass{pracamgr}

\usepackage{polski}

%Jesli uzywasz kodowania polskich znakow ISO-8859-2 nastepna linia powinna byc 
%odkomentowana
\usepackage[latin2]{inputenc}
%Jesli uzywasz kodowania polskich znakow CP-1250 to ta linia powinna byc 
%odkomentowana
%\usepackage[cp1250]{inputenc}

% Dane magistranta:

\author{Julia Herman-Iżycka, Marcin Jedynak, Jakub Partyka, Aleksandra Skrzypczak}

\nralbumu{280468, 292629, 236188, 234586}

\title{Goodle - platforma e-learningowa}

\tytulang{Google - e-learning platform}

%kierunek: Matematyka, Informatyka, ...
\kierunek{Informatyka}

% informatyka - nie okreslamy zakresu (opcja zakomentowana)
% matematyka - zakres moze pozostac nieokreslony,
% a jesli ma byc okreslony dla pracy mgr,
% to przyjmuje jedna z wartosci:
% {metod matematycznych w finansach}
% {metod matematycznych w ubezpieczeniach}
% {matematyki stosowanej}
% {nauczania matematyki}
% Dla pracy licencjackiej mamy natomiast
% mozliwosc wpisania takiej wartosci zakresu:
% {Jednoczesnych Studiow Ekonomiczno--Matematycznych}

% \zakres{Tu wpisac, jesli trzeba, jedna z opcji podanych wyzej}

% Praca wykonana pod kierunkiem:
% (podaæ tytu³/stopieñ imiê i nazwisko opiekuna
% Instytut
% ew. Wydzia³ ew. Uczelnia (je¿eli nie MIM UW))
\opiekun{dr Jacek Sroka\\
  Instytut Informatyki\\
  }

% miesi±c i~rok:
\date{Maj 2012}

%Podaæ dziedzinê wg klasyfikacji Socrates-Erasmus:
\dziedzina{ 
%11.0 Matematyka, Informatyka:\\ 
%11.1 Matematyka\\ 
%11.2 Statystyka\\ 
11.3 Informatyka\\ 
%11.4 Sztuczna inteligencja\\ 
%11.5 Nauki aktuarialne\\
%11.9 Inne nauki matematyczne i informatyczne
}

%Klasyfikacja tematyczna wedlug AMS (matematyka) lub ACM (informatyka)
\klasyfikacja{H.3.5 Online information services\\

}

% S³owa kluczowe:
\keywords{Goodle, platforma edukacyjna, e-learning}

% Tu jest dobre miejsce na Twoje w³asne makra i~¶rodowiska:
\newtheorem{defi}{Definicja}[section]

% koniec definicji

\begin{document}
\maketitle

%
%  Abstrakt
%
\begin{abstract}
Dokument opisuje, stworzoną w ramach projektu licencjackiego, platformę 
edukacyjną Goodle. Nazwa stanowi kontaminację słów Google i Moodle. 
Innowacyjne technologie i narzędzia Google'a wykorzystano do stworzenia
programu. Moodle, natomiast, to używana obecnie przez Uniwersytet Warszawski
platforma-rywal, w stosunku do której Goodle stanowi alternatywę. Integracja z
systemem USOS zapewnia aplikacji ważną, lecz nie jedyną, przewagę. 

Grono potencjalnych odbiorców stanowią studenci i pracownicy akademiccy. Nie 
stworzono dotychczas produktu zapewniającego wszystkie pożądane funkcje, co 
stanowiło motywację do realizacji projektu. 

W kolejnych rozdziałach pracy opisane zostaną: przyczyny stworzenia systemu,
charakterystyka potencjalnych użytkowników, funkcjonalność i architektura 
aplikacji, możliwe scenariusze użycia, zastosowana przy tworzeniu metodyka,
narzędzia i technologie oraz ewentualne przyszłe rozszerzenia.
\end{abstract}

\tableofcontents
%\listoffigures
%\listoftables

\chapter*{Wprowadzenie} % Ta gwiazdka jest potrzebna, czy to błąd? 
\addcontentsline{toc}{chapter}{Wprowadzenie} % Co robi ta komenda?

%
%  Wstęp. Jak wyglądała sytuacja przed rozpoczęciem projektu i dlaczego
%  powstała potrzeba stworzenia aplikacji.
%
\section{Sformułowanie problemu}

%
%  Definicje użytych w dokumencie trudnych terminów i rozwinięcie akronimów.
%
\section{Definicje}
\begin{description}
   \item[E-learning]
      nauczanie z wykorzystaniem sieci komputerowych i internetu.
   \item[Framework]
      szkielet do budowy aplikacji, definiuje podstawowy mechanizm działania
      programu i dostarcza zestaw komponentów i bibliotek ogólnego
      przeznaczenia do wykonywania określonych zadań. 
   \item[GAE]
      Google App Engine, działająca na zasadzie chmury platforma do tworzenia 
      aplikacji internetowych i utrzymywania ich w centrach danych Google.
   \item[GWT]
      Google Web Toolkit, framework i zestaw narzędzi umożliwiający 
      i ułatwiający użytkownikom tworzenie aplikacji internetowych.
   \item[Lama]
      udomowiony ssak parzystokopytny z rodziny wielbłądowatych, do którego
      podejrzaną słabość czuł jeden z uczestników projektu.
   \item[Moodle]
      Modular Object-Oriented Dynamic Learning Environment, platforma 
      e-learningowa. 
   \item[USOS]
      Uniwersytecki System Obsługi Studiów.
\end{description}

%
%  Opisana pokrótce zawartość każdego rozdziału.
%
\section{Omówienie zawartości}

Dokument podzielony został na 5 rozdziałów, z których każdy zawiera po kilka
logicznie powiązanych sekcji:

%  TODO Uzupełnić nazwę rozdziału
W rozdziale \emph{ } objaśniono przyczyny rozpoczęcia projektu i
uzasadniono wybrany scenariusz biznesowy. Opisano najpierw jakie braki i wady 
istniejącego rozwiązania - platformy Moodle - stały się motywacją pracy. 
Następnie wskazano jakie funkcje powinien zapewniać alternatywny, idealny
system. Dalej streszczono rozważane opcje biznesowe i zaprezentowano argumenty,
które zadecydowały o stworzeniu nowej aplikacji. Na koniec przedstawiono krótką
charakterystykę potencjalnych użytkowników.

%  TODO Uzupełnić po napisaniu rozdziału
Rozdział \emph{Możliwości systemu} omawia dokładnie funkcjonalność produktu.

%  TODO Uzupełnić po napisaniu rozdziału
W kolejnym rozdziale omówiono \emph{możliwe scenariusze użycia}. 

%  TODO Uzupełnić po napisaniu rozdziału
Następnie dokument przechodzi do opisu \emph{użytych narzędzi i technologii}. 

%  TODO Uzupełnić po napisaniu rozdziału
\emph{Metodyka pracy} 

%  TODO Uzupełnić po napisaniu rozdziału
\emph{Podsumowanie}

%
%  Przyczyny rozpoczęcia projektu i uzasadnienie wybranego scenariusza
%  biznesowego.
%
\chapter{ } % TODO Uzupełnić nazwę rozdziału

%
%  Dlaczego zdecydowano się na rozpoczęcie projektu.
%
\section{Motywacja}

Od momentu uruchomienia system USOS podlega kolejnym usprawnieniom. Niestety, 
wprowadzane modyfikacje i dodawane moduły nie są w stanie nadążyć za rosnącymi 
wymaganiami użytkowników.
 
Jednym z oczekiwań niektórych odbiorców jest integracja serwisu z kontem Google i 
oferowanymi przez firmę narzędziami. Dla wielu użytkowników uciążliwy jest również 
brak konsolidacji USOS z innymi portalami akademickimi np. platformą Moodle. 
Tworzenie kursów, zarządzanie użytkownikami czy umieszczanie wyników studentów
musi się zatem odbywać na kilku stronach w sposób niezależny.

Do problemów Moodle należy np. trudny w obsłudze, nieprzyjazny dla użytkownika 
interfejs. Spotyka się on z nieustanną krytyką ze strony studentów i pracowników 
wydziału. Dodatkowo platforma nie zapewnia prowadzącemu zajęcia dostatecznej 
elastyczności i łatwości w udostępnianiu i organizacji udostępnianych materiałów. 
	
Wynikiem wymienionych mankamentów jest niezadowolenie użytkowników oraz 
nieefektywne wykorzystanie potencjału tkwiącego w tego typu portalach. Celem 
postawionym przed nowym systemem było rozwiązanie ich, co uatrakcyjniłoby pracę 
i zaoszczędziło czas zarówno pracowników, jak i studentów oraz podniosło komfort 
kształcenia na Uniwesytecie Warszawskim.

%
%  Jakie powinny być cechy idealnego produktu, czyli do czego dążył projekt.
%
\section{Cele}

Najważniejszym celem, jaki stawiano sobie podczas realizacji systemu, było
zapewnienie prostego, intuicyjnego interfejsu. Moodle oferował użytkownikom,
szczególnie od strony prowadzącego, rozmaite funkcje, np. podczas umieszczania
nowych materiałów. Niestety powodowało to przeładowanie aplikacji licznymi
przyciskami i opcjami. Zapewniana elastyczność i szeroki wachlarz możliwości
nie skusiły użytkowników, którzy większości funkcji nie używali, a często
nawet nie znali. Pisząc Goodle, przyjęto inną filozofię. Ustalono listę
pożądanych przez odbiorców opcji, w tym kluczowych do uczestnictwa w kursach
i administrowania nimi, i postanowiono zapewnić to minimum w jak najbardziej
klarownej formie. 

Drugi ważny cel, jaki postawili sobie twórcy, stanowiła integracja
tworzonego portalu z systemem USOS. Kiedy rozpoczęto projekt, nie było
możliwe zalogowanie się do Moodle przy pomocy uczelnianego Centralnego
Systemu Uwierzytelniania. Opcję tę dodano później. Dodatkowo tworzone na
platformie e-learningowej kursy były zwykle odpowiednikami tych z USOS. Mimo
to, nie istniała możliwość transferu infromacji np. danych kursu, listy
użytkowników czy wystawionych ocen. W Goodle postanowiono zapewnić taką 
komunikację. Niestety te zamierzenia udało się zrealizować tylko
częściowo, z uwagi za ograniczenia dostępnego USOS API.   

%
%  Jakie alternatywne rozwiązania rozpatrywano i dlaczego zdecydowano się
%  na stworzenie własnej aplikacji.
%
%  TODO Poniższa sekcja jest laniem wody na pełną skalę. Skomponowałem
%  ją z materiałów wygrzebanych z wizji, ale może warto z niej zrezygnować.

%\section{Odrzucone scenariusze} 
%
%Praca z systemem Moodle jest żmudna i nieefektywna z powodu problemów 
%wymienionych we wcześniejszych częściach dokumentu. Rozsądnym scenariuszem
%było zatem przejście na alternatywną platformę e-learningową. Rozwiązania 
%komercyjne, np. Blackboard Learning, trzeba było niestety odrzucić ze względu
%na koszty. Drugim problemem okazała się synchronizacja takiego systemu z
%USOS. W przypadku rozwiązania open-source (np. platformy OLAT) wymagałaby
%dokonania pracochłonnych modyfikacji kodu. Z tego powodu, zdecydowano się na 
%stworzenie nowej aplikacji, dostosowanej do potrzeb studentów i pracowników 
%Uniwersytetu Warszawskiego. 

%
%  Opis potencjalnych odbiorców i ich wymagań w stosunku do systemu, które
%  postanowiono spełnić.
%
\section{Charakterystyka użytkowników}

Podczas tworzenia aplikacji przypisano potencjalnych użytkowników
do jednej z czterech ról, których wymagania następnie poddano analizie:

\begin{itemize}
   \item Administratorzy
      \begin{description}
         \item[Opis] Osoby zarządzające serwisem i dbające o jego bezpieczeństwo.
         \item[Potrzeby] Interfejs zapewniający wygodne narzędzia do 
            administracji systemem, pozwalający dokonywać typowych operacji 
            przez przeglądarkę.
      \end{description}
   \item Studenci
      \begin{description}
         \item[Opis] Studenci zapisani na kursy prowadzone przy użyciu platformy.
         \item[Potrzeby] Intuicyjny i prosty w obsłudze interfejs zapewniający: 
            wygodny dostęp do materiałów, możliwość przesłania prac domowych 
            prowadzącemu, wgląd w wyniki kursu; przejrzysta szata graficzna, 
            integracja z serwisami USOS. 
      \end{description}  
   \item Byli studenci
      \begin{description}
         \item[Opis] Osoby, zapisane na kursy prowadzone przy użyciu platformy, 
            które przestały być studentami.
         \item[Potrzeby] Intuicyjny i prosty w obsłudze interfejs zapewniający: 
            wygodny dostęp do materiałów oraz wgląd w wyniki kursu; przejrzysta 
            szata graficzna.
      \end{description}  
   \item Prowadzący zajęcia
      \begin{description}
         \item[Opis] Pracownicy Uniwersytetu Warszawskiego prowadzący zajęcia 
            przy użyciu platformy.
         \item[Potrzeby] Intuicyjny i prosty w obsłudze interfejs zapewniający 
            możliwość: ograniczenia dostępu do kursu wybranej grupie
            studentów, publikacji materiałów i wyników kursu, zadania i 
            przyjęcia prac domowych; integracja z serwisami USOS, przejrzysta 
            szata graficzna.
      \end{description}  
\end{itemize}

%
%  Możliwości programu. Poszczególne funkcje powinny być sekcjami tego
%  rozdziału.
%
\chapter{Możliwości systemu}

Obecnie aplikacja Goodle oferuje użytkownikom następujące funkcje:

\section{Tworzenie i administracja kursem, synchronizacja z USOS}

Każdy użytkownik Goodle ma możliwość utworzenia zajęć poprzez kliknięcie
przycisku "Stwórz kurs" w lewym panelu. Powstawowymi operacjami na 
utworzonym kursie są zmiana jego nazwy oraz semestru i roku szkolnego, 
w którym odbywają się zajęcia. Kurs w Goodle, podobnie jak na Uniwersytecie
Warszawskim może odbywać się w semestrze zimowym, letnim lub być całoroczny.
Domyślnie kurs zostaje przypisany do aktualnego semestru.
Zakładka "Informacje" umożliwia dodatkowo dodanie opisu kursu oraz związanej
z nim bibliografii.

Zdaża się często, że tworzony kurs posiada swój odpowiednik w USOS. W takim
przypadku, zamiast wpisywać dane ręcznie, prowadzący ma możliwość pobrania
ich z systemu.

\section{Wyszukiwanie kursów}

Możliwe jest wyszukiwanie zajęć używając panelu znajdującego
się u góry strony. W trakcie wpisywania, aplikacja wspiera użytkownika,
podpowiadając pasujące nazwy. 

Na liście wyników wyszukiwania podana jest nie tylko nazwa kursu, 
ale również jego semestr i rok szkolny, co pozwala łatwo odróżnić zajęcia z 
danego przedmiotu prowadzone w kolejnych latach.  

\section{Rejestracja na kurs i zarządzanie użytkownikami}

Aby prowadzenie zajęć miało sens, do kursu powinni dołączyć uczestnicy. 
Nowoutworzony kurs nie jest ukryty - użytkownicy mogą go wyszukać i obejrzeć
informacje na jego temat, jednak aby uzyskać wgląd w dodane materiały, prace
domowe czy listę uczestników, konieczne jest pomyślne przejście procesu 
rejestracji.

Istnieją dwa poziomy zabezpieczenia kursu: otwarty i zabezpieczony hasłem. 
Na zajęcia użytkownik Goodle może zarejestrować się, klikając przycisk 
"zarejestruj", znajdujący się obok nazwy kursu. W przypadku zajęć otwartych
spowoduje to natychmiastowe dołączenie użytkownika do listy uczęszczających.
Jeżeli prowadzący zdefiniował hasło, do ukończenia rejestracji konieczne będzie
podanie go. Lista kursów, w których użytkownik uczestniczy, podobnie
jak prowadzonych przez niego, pojawia się w panelu po lewej stronie ekranu.

Domyślnie kurs nie jest chroniony hasłem, ale prowadzący ma możliwość zmiany
polityki w dowolnym momencie. 

Wszyscy zarejestrowani mają możliwość obejrzenia listy współuczestników, 
dostępnej w zakładce "Uczestnicy". Dodatkowo prowadzący może
wyrejestrować wybranych uczestników z kursu, zaznaczając ich na wymienionej
liście, a następnie klikając "usuń". 

\section{Dodawanie i przeglądanie materiałów}

Prowadzący kurs dysponuje możliwością umieszczania na stronie kursu 
materiałów - tekstu zawierającego na przykład streszczenie wykładu czy
dodatkowe informacje przeznaczone dla bardziej zaangażowanych uczestników.
Każdy materiał może być widoczny dla uczestników lub ukryty - widoczny
tylko dla prowadzącego. Umożliwia do wcześniejsze przygotowanie informacji
i udostępnianie ich studentom w dogodnym momencie. Dodatkowo materiały mogą 
zostać wzbogacone załadowanymi na serwer plikami, np.
dokumentami pdf, obrazkami czy filmami.  

Materiały modyfikowane są w trybie edycji, do której prowadzący przechodzi
klikając "edytuj" w zakładce "Materiały". Od tego momentu ma możliwość
dodawania i usuwania materiałów, zmieniania ich nazwy, treści, widoczoności
oraz ładowania plików. Zmiany zostaną zapisane dopiero, kiedy prowadzący
zatwierdzi je klikając "zapisz", dlatego też w każdej chwili może 
wycofać się z wprowadzonych modyfikacji wybierając "anuluj".

\section{Zadawanie i ocena prac domowych}

Aby sprawdzić umiejętności uczestników i zachęcić ich do regularnej pracy,
prowadzący może dawać im zadania do wykonania w zakładce "Zadania". 
W wielu aspektach prace domowe funkcjonują podobnie jak materiały: 
treść zadania może zostać wzbogacona dodatkowymi plikami, zadanie może być
widoczne dla wszystkich lub tylko dla prowadzącego oraz modyfikacji
zadań dokonuje się w specjalnym trybie edycji, w którym prowadzący może
w każdym momencie wycofać się z wprowadzonych zmian. Dodatkowo, prowadzący
może ustawić termin, do którego należy przysyłać rozwiązanie problemu.

Na specjalnej liście prowadzący widzi kto z uczestników umieścił plik
i czy zrobił to w terminie. Może przejrzeć zamieszczone rozwiązania 
i ocenić je. Uzyskany wynik studenci zobaczą obok zadania.

Każde dodane, odkryte zadanie jest widoczne dla studentów nie tylko w zakładce
"Zadania", ale również w prawym panelu wraz z terminem nadsyłania prac, 
gdzie użytkownicy widzą wszystkie prace domowe z kursów, na które są
zarejestrowani.

%
%  Przypadki użycia. Ważniejsze przypadki powinny być sekcjami tego
%  rozdziału. Ewentualne zdjęcia interfejsu. 
%
\chapter{Możliwe scenariusze użycia}


%
%  Użyte narzędzia, biblioteki, technologie. Poszczególne rozwiązania powinny 
%  być sekcjami tego rozdziału. Ewentualne zdjęcia.
%
\chapter{Zastosowane technologie}

\section{Java}
\section{Git}
\section{Google Web Toolkit}
\section{Google App Engine}
\section{Selenium}

%
%  Sposób pracy. Epicki epos o bohaterach, opis heroicznego wysiłku, potu, krwi 
%  i łez. Rozdział powinien być pisany heksametrem.
%
\chapter{Metodyka pracy}

%
%  W jakich iteracjach pracowano, jak wyglądała komunikacja i podział
%  zadań, jakie panowały zasady, kto był szefem projektu :)
%
\section{System pracy}

%
%  Jakie dokumenty powstały, ich cel i zawartość.
%
\section{Dokumentacja}


%
%  Ocena serwisu, wnioski (poprawa systemu pracy, uniknięcie błędów 
%  w przyszłości).
%
\chapter{Podsumowanie}

%
%  Jak mógłby dalej rozwijać się produkt. Czego nie udało się napisać i 
%  nigdy nie zostanie napisane. Zdjęcie członków zespołu nad morzem, używających
%  zdobytych dyplomów jako podkładek pod gofry. 
%
\section{Możliwe rozszerzenia}


\begin{thebibliography}{99}
\addcontentsline{toc}{chapter}{Bibliografia}


%Przykład wpisu do bibliografii
%\bibitem[Bea65]{beaman} Juliusz Beaman, \textit{Morbidity of the Jolly
%    function}, Mathematica Absurdica, 117 (1965) 338--9.



\end{thebibliography}

\end{document}


%%% Local Variables:
%%% mode: latex
%%% TeX-master: t
%%% coding: latin-2
%%% End:
